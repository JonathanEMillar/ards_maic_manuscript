% Options for packages loaded elsewhere
\PassOptionsToPackage{unicode}{hyperref}
\PassOptionsToPackage{hyphens}{url}
\PassOptionsToPackage{dvipsnames,svgnames,x11names}{xcolor}
%
\documentclass[
  11,
  a4paper,
]{article}

\usepackage{amsmath,amssymb}
\usepackage{setspace}
\usepackage{iftex}
\ifPDFTeX
  \usepackage[T1]{fontenc}
  \usepackage[utf8]{inputenc}
  \usepackage{textcomp} % provide euro and other symbols
\else % if luatex or xetex
  \usepackage{unicode-math}
  \defaultfontfeatures{Scale=MatchLowercase}
  \defaultfontfeatures[\rmfamily]{Ligatures=TeX,Scale=1}
\fi
\usepackage{lmodern}
\ifPDFTeX\else  
    % xetex/luatex font selection
  \setmainfont[Numbers=Lowercase,Numbers=Proportional]{Times New Roman}
\fi
% Use upquote if available, for straight quotes in verbatim environments
\IfFileExists{upquote.sty}{\usepackage{upquote}}{}
\IfFileExists{microtype.sty}{% use microtype if available
  \usepackage[]{microtype}
  \UseMicrotypeSet[protrusion]{basicmath} % disable protrusion for tt fonts
}{}
\makeatletter
\@ifundefined{KOMAClassName}{% if non-KOMA class
  \IfFileExists{parskip.sty}{%
    \usepackage{parskip}
  }{% else
    \setlength{\parindent}{0pt}
    \setlength{\parskip}{6pt plus 2pt minus 1pt}}
}{% if KOMA class
  \KOMAoptions{parskip=half}}
\makeatother
\usepackage{xcolor}
\usepackage[top=15mm,left=22.5mm,right=22.5mm,bottom=15mm]{geometry}
\setlength{\emergencystretch}{3em} % prevent overfull lines
\setcounter{secnumdepth}{-\maxdimen} % remove section numbering
% Make \paragraph and \subparagraph free-standing
\ifx\paragraph\undefined\else
  \let\oldparagraph\paragraph
  \renewcommand{\paragraph}[1]{\oldparagraph{#1}\mbox{}}
\fi
\ifx\subparagraph\undefined\else
  \let\oldsubparagraph\subparagraph
  \renewcommand{\subparagraph}[1]{\oldsubparagraph{#1}\mbox{}}
\fi


\providecommand{\tightlist}{%
  \setlength{\itemsep}{0pt}\setlength{\parskip}{0pt}}\usepackage{longtable,booktabs,array}
\usepackage{calc} % for calculating minipage widths
% Correct order of tables after \paragraph or \subparagraph
\usepackage{etoolbox}
\makeatletter
\patchcmd\longtable{\par}{\if@noskipsec\mbox{}\fi\par}{}{}
\makeatother
% Allow footnotes in longtable head/foot
\IfFileExists{footnotehyper.sty}{\usepackage{footnotehyper}}{\usepackage{footnote}}
\makesavenoteenv{longtable}
\usepackage{graphicx}
\makeatletter
\def\maxwidth{\ifdim\Gin@nat@width>\linewidth\linewidth\else\Gin@nat@width\fi}
\def\maxheight{\ifdim\Gin@nat@height>\textheight\textheight\else\Gin@nat@height\fi}
\makeatother
% Scale images if necessary, so that they will not overflow the page
% margins by default, and it is still possible to overwrite the defaults
% using explicit options in \includegraphics[width, height, ...]{}
\setkeys{Gin}{width=\maxwidth,height=\maxheight,keepaspectratio}
% Set default figure placement to htbp
\makeatletter
\def\fps@figure{htbp}
\makeatother
% definitions for citeproc citations
\NewDocumentCommand\citeproctext{}{}
\NewDocumentCommand\citeproc{mm}{%
  \begingroup\def\citeproctext{#2}\cite{#1}\endgroup}
\makeatletter
 % allow citations to break across lines
 \let\@cite@ofmt\@firstofone
 % avoid brackets around text for \cite:
 \def\@biblabel#1{}
 \def\@cite#1#2{{#1\if@tempswa , #2\fi}}
\makeatother
\newlength{\cslhangindent}
\setlength{\cslhangindent}{1.5em}
\newlength{\csllabelwidth}
\setlength{\csllabelwidth}{3em}
\newenvironment{CSLReferences}[2] % #1 hanging-indent, #2 entry-spacing
 {\begin{list}{}{%
  \setlength{\itemindent}{0pt}
  \setlength{\leftmargin}{0pt}
  \setlength{\parsep}{0pt}
  % turn on hanging indent if param 1 is 1
  \ifodd #1
   \setlength{\leftmargin}{\cslhangindent}
   \setlength{\itemindent}{-1\cslhangindent}
  \fi
  % set entry spacing
  \setlength{\itemsep}{#2\baselineskip}}}
 {\end{list}}
\usepackage{calc}
\newcommand{\CSLBlock}[1]{\hfill\break\parbox[t]{\linewidth}{\strut\ignorespaces#1\strut}}
\newcommand{\CSLLeftMargin}[1]{\parbox[t]{\csllabelwidth}{\strut#1\strut}}
\newcommand{\CSLRightInline}[1]{\parbox[t]{\linewidth - \csllabelwidth}{\strut#1\strut}}
\newcommand{\CSLIndent}[1]{\hspace{\cslhangindent}#1}

\usepackage{lscape}
\newcommand{\blandscape}{\begin{landscape}}
\newcommand{\elandscape}{\end{landscape}}
\pagenumbering{roman}
\makeatletter
\@ifpackageloaded{caption}{}{\usepackage{caption}}
\AtBeginDocument{%
\ifdefined\contentsname
  \renewcommand*\contentsname{Table of contents}
\else
  \newcommand\contentsname{Table of contents}
\fi
\ifdefined\listfigurename
  \renewcommand*\listfigurename{List of Figures}
\else
  \newcommand\listfigurename{List of Figures}
\fi
\ifdefined\listtablename
  \renewcommand*\listtablename{List of Tables}
\else
  \newcommand\listtablename{List of Tables}
\fi
\ifdefined\figurename
  \renewcommand*\figurename{Supplementary-Figure}
\else
  \newcommand\figurename{Supplementary-Figure}
\fi
\ifdefined\tablename
  \renewcommand*\tablename{Table}
\else
  \newcommand\tablename{Table}
\fi
}
\@ifpackageloaded{float}{}{\usepackage{float}}
\floatstyle{ruled}
\@ifundefined{c@chapter}{\newfloat{codelisting}{h}{lop}}{\newfloat{codelisting}{h}{lop}[chapter]}
\floatname{codelisting}{Listing}
\newcommand*\listoflistings{\listof{codelisting}{List of Listings}}
\makeatother
\makeatletter
\makeatother
\makeatletter
\@ifpackageloaded{caption}{}{\usepackage{caption}}
\@ifpackageloaded{subcaption}{}{\usepackage{subcaption}}
\makeatother
\ifLuaTeX
  \usepackage{selnolig}  % disable illegal ligatures
\fi
\usepackage{bookmark}

\IfFileExists{xurl.sty}{\usepackage{xurl}}{} % add URL line breaks if available
\urlstyle{same} % disable monospaced font for URLs
\hypersetup{
  pdftitle={The genomic landscape of Acute Respiratory Distress Syndrome: a meta-analysis by information content of whole-genome studies of the host response.},
  colorlinks=true,
  linkcolor={blue},
  filecolor={Maroon},
  citecolor={Blue},
  urlcolor={Blue},
  pdfcreator={LaTeX via pandoc}}

\title{The genomic landscape of Acute Respiratory Distress Syndrome: a
meta-analysis by information content of whole-genome studies of the host
response.}
\author{}
\date{}

\begin{document}
\maketitle

\setstretch{1.5}
\begin{center}

\textbf{\LARGE Supplementary Material}

\end{center}

\newpage

\subsection{Supplementary Methods}\label{supplementary-methods}

Search strategy

Inclusion criteria

Glossary

\newpage

\subsubsection{Search Strategy}\label{search-strategy}

We used the following strategy to search MEDLINE and a direct
translation to search Embase.

\textbf{1} exp Respiratory Distress Syndrome, Adult/

\textbf{2} ``acute lung injury*``.ti,ab,kf,kw

\textbf{3} 1 OR 2

\textbf{4} ``gene*``.mp

\textbf{5} ``genome*``.mp

\textbf{6} ``transcript*``.mp

\textbf{7} ``protein*``.mp

\textbf{8} 4 OR 5 OR 6 OR 7

\textbf{9} 3 AND 8

\textbf{10} (``COVID-19*'' OR ``COVID19*'' OR ``COVID-2019*'' OR
``covid'').ti,ab,kf,kw

\textbf{11} (``SARS-CoV-2*'' OR ``SARSCov-2*'' OR ``SARSCoV2*'' OR
``SARS-CoV2'').ti,sh,kf,kw

\textbf{12} (``2019-nCoV*'' OR ``2019nCoV*'' OR ``19- nCoV*'' OR
``19nCoV*'' OR ``nCoV2019*'' OR ``nCoV-2019*'' OR ``nCoV19*'' OR ``nCoV-
19*``).ti,ab,kf,kw

\textbf{13} 10 OR 11 OR 12

\textbf{14} 9 NOT 13

\textbf{15} Letter.pt OR Conference Abstract.pt OR Conference Paper.pt
OR Conference Review.pt OR Editorial.pt OR Erratum.pt OR Review.pt OR
Note.pt OR Tombstone.pt

\textbf{16} 14 NOT 15

\textbf{17} exp *adolescence/ or exp *adolescent/ or exp *child/ or exp
*childhood disease/ or exp *infant disease/ or (adolescen* or babies or
baby or boy? or boyfriend or boyhood or girlfriend or girlhood or child
or child* or child*3 or children* or girl? or infan* or juvenil* or
juvenile* or kid? or minors or minors* or neonat* or neo-nat* or
newborn* or new-born* or paediatric* or peadiatric* or pediatric* or
perinat* or preschool* or puber* or pubescen* or school* or teen* or
toddler? or underage? or under-age? or youth*).ti,kw

\textbf{18} 16 NOT 17

\textbf{19} ((exp animal/ or nonhuman/) NOT exp human/)

\textbf{20} 18 NOT 19

\textbf{21} limit 20 to yr=``1967-Current''

\newpage

\subsubsection{Inclusion criteria}\label{inclusion-criteria}

Inclusion:

\begin{itemize}
\tightlist
\item
  Human studies: \emph{in-vivo} or \emph{in-vitro}
\item
  Adults (age ≥ 18 years)
\item
  Acute Respiratory Distress Syndrome (ARDS)

  \begin{itemize}
  \tightlist
  \item
    by any contemporaneous definition
  \end{itemize}
\item
  Accepted methodologies:

  \begin{itemize}
  \tightlist
  \item
    CRISPR screen
  \item
    RNAi screen
  \item
    Protein-protein interaction study
  \item
    Host proteins incorporated into virion or virus-like particle
  \item
    Genome wide association study
  \item
    Transcriptomic study
  \item
    Proteomic study
  \end{itemize}
\end{itemize}

Exclusion:

\begin{itemize}
\tightlist
\item
  Children (age \textless{} 18 years)
\item
  Animal studies
\item
  Meta-analyses, \emph{in-silico} analyses, or re-analysis of previously
  published data
\item
  Excluded methodologies:

  \begin{itemize}
  \tightlist
  \item
    \emph{In-vitro} human studies simulating ARDS
  \item
    Candidate \emph{in-vivo} or \emph{in-vitro} transcriptomic or
    proteomic studies (defined as those investigating \textless{} 50
    genes)
  \item
    Candidate gene association studies
  \item
    Studies including fewer than 5 individuals in either the control or
    ARDS arm
  \end{itemize}
\end{itemize}

\newpage

\subsubsection{Glossary}\label{glossary}

\textbf{MAIC score} - the score assigned by MAIC to a given gene
considering all lists.

\textbf{Gene score} - the score assigned by MAIC to a given gene in a
given list.

\textbf{Total MAIC score} - the sum of all scores assigned by MAIC to a
genes in a given list.

\textbf{Contributing total MAIC score} - the sum of all scores assigned
by MAIC to a genes in a given list where that score contributes to the
MAIC score for that gene (i.e., excluding those gene scores that are not
used because a gene score from another list in the same category is
greater).

\newpage

\subsection{Supplementary Results}\label{supplementary-results}

Supplementary Figures 1-7

Supplementary Tables 1-4

\begin{figure}[H]

{\centering \includegraphics{../img/Supplementary_Figure_1.png}

}

\caption{\textbf{Systematic review inclusion diagram}. Abbreviations: db
- data base; GEO - NCBI Gene Expression Omnibus.}

\end{figure}%%
\begin{figure}[H]

{\centering \includegraphics{../img/Supplementary_Figure_2.png}

}

\caption{\textbf{Attributing information in MAIC}. (a) Shared
information between gene lists. Links indicate shared summed common gene
scores between studies. (b) Proportion of replicated genes. Circle
diameter is equal to logarithm (base 2) of gene number per list. (c)
Total MAIC score (totMS) normalised by number of genes. Overlapping
circles denote equal normalised totMS and contribution (ctotMS - sum of
common gene scores contributing to MAIC score for a gene), indicating
all gene scores contributed to MAIC.}

\end{figure}%%
\begin{figure}[H]

{\centering \includegraphics{../img/Supplementary_Figure_3.png}

}

\caption{\textbf{Overlap between ARDS MAIC and ARDS-associated genes and
ARDS MAIC and coronavirus MAIC}. (a) Euler diagram of gene overlap
between ARDS MAIC and a BioLitMine search using the ARDS MeSH term. (b)
Schematic overview of a co-expression search for genes identified in the
BioLitMine search but not present in ARDS MAIC and a stacked bar plot of
the proportion of the 100 most co-expressed genes of this group and ARDS
MAIC. (c) Euler diagram of gene overlap between ARDS MAIC and the ARDS
Database of Genes. (d) Euler diagram of gene overlap between ARDS MAIC
and a MAIC of COVID-19 host-response studies. (e) Heatmap of the 50 top
ranked ARDS MAIC genes also prioritised by the coronavirus MAIC,
displaying the ARDS MAIC score for each gene, highest gene score in each
category, and the number of supporting gene lists.}

\end{figure}%%
\begin{figure}[H]

{\centering \includegraphics{../img/Supplementary_Figure_4.png}

}

\caption{\textbf{Tissue and cell-specific expression}. (a) Bar plot of
the tissue type in which genes are identified - all genes (n=7,085). (b)
Bar plot of the tissue type in which genes are identified - prioritised
genes (n=1,306). (c) Bar plot of the proportion of genes identified
solely in blood meeting mRNA expression thresholds in bulk lung tissue.
nTPM - normalised transcripts per million. (d) Heatmap of mRNA
expression in lung cell-types for genes identified in studies based on
airways sampling. (e) Heatmap of mRNA expression in blood cell-types for
genes identified solely in studies based on blood sampling. (f)
Manhatten plot of the top 20 cell types overenriched for expression of
genes identified by studies based on airways sampling. (g) Manhatten
plot of the top 20 cell types overenriched for expression of genes
identified by studies based on blood sampling.}

\end{figure}%%
\begin{figure}[H]

{\centering \includegraphics{../img/Supplementary_Figure_5.png}

}

\caption{\textbf{Functional enrichment}. (a) Significantly enriched KEGG
terms (\emph{P} \textless{} 0.01) for prioritised genes. Terms size
proportional to recall. (b) Significantly enriched WikiPathways terms
(\emph{P} \textless{} 0.01) for prioritised genes. Terms size
proportional to recall. (c) Scatter plot of the semantic similarity
between signficantly enriched GO cellular component terms for
prioritised genes (d) Manhatten plot of the overenrichment of
prioritised genes against the GWAS catolog.}

\end{figure}%%
\begin{figure}[H]

{\centering \includegraphics{../img/Supplementary_Figure_6.png}

}

\caption{\textbf{PPI clusters}. A protein-protein interaction network of
prioritsed genes and the 10 largest graph-based clusters. Functional
annotation by hand based on a concencus of enriched Reactome, KEGG,
WikiPathways, and GO Biological Process terms.}

\end{figure}%%
\begin{figure}[H]

{\centering \includegraphics{../img/Supplementary_Figure_7.png}

}

\caption{\textbf{Details of MAIC on sub-groups}. (a) Gene prioritisation
for the ARDS MAIC ARDS vs.~non-ARDS controls sub-group using the Unit
Invariant Knee method. Intersection of lines identifies elbow point of
best-fit curve. 130 genes in upper left quadrant were prioritied. (b)
Euler diagrams of gene overlap between the ARDS vs.~non-ARDS controls
sub-group and a BioLitMine search using the ARDS MeSH term and the ARDS
Database of Genes. (c) Shared information between ARDS vs.~non-ARDS
controls gene lists. Links indicate shared summed common gene scores
between studies. (d) Gene prioritisation for the ARDS MAIC survival
sub-group using the Unit Invariant Knee method. Intersection of lines
identifies elbow point of best-fit curve. 33 genes in upper left
quadrant were prioritied. (e) Euler diagrams of gene ovelap between the
survival sub-group and a BioLitMine search using the ARDS MeSH term and
the ARDS Database of Genes. (f) Shared information between survival gene
lists. Links indicate shared summed common gene scores between studies.}

\end{figure}%

\newpage

\textbf{Supplementary Table 1. Gene list information content and
contribution.}

\begin{longtable}[]{@{}
  >{\raggedright\arraybackslash}p{(\columnwidth - 10\tabcolsep) * \real{0.2719}}
  >{\raggedright\arraybackslash}p{(\columnwidth - 10\tabcolsep) * \real{0.1491}}
  >{\raggedright\arraybackslash}p{(\columnwidth - 10\tabcolsep) * \real{0.1228}}
  >{\raggedright\arraybackslash}p{(\columnwidth - 10\tabcolsep) * \real{0.1140}}
  >{\raggedright\arraybackslash}p{(\columnwidth - 10\tabcolsep) * \real{0.1667}}
  >{\raggedright\arraybackslash}p{(\columnwidth - 10\tabcolsep) * \real{0.1754}}@{}}
\toprule\noalign{}
\begin{minipage}[b]{\linewidth}\raggedright
\textbf{Study}
\end{minipage} & \begin{minipage}[b]{\linewidth}\raggedright
\textbf{Method}
\end{minipage} & \begin{minipage}[b]{\linewidth}\raggedright
\textbf{Category}
\end{minipage} & \begin{minipage}[b]{\linewidth}\raggedright
\textbf{N genes}
\end{minipage} & \begin{minipage}[b]{\linewidth}\raggedright
\textbf{totMS} (\% sum)
\end{minipage} & \begin{minipage}[b]{\linewidth}\raggedright
\textbf{ctotMS} (\% sum)
\end{minipage} \\
\midrule\noalign{}
\endhead
\bottomrule\noalign{}
\endlastfoot
Sarma\textsuperscript{1} & Transcriptomics & RNA-seq & 4954 & 50.8 &
53.1 \\
Juss\textsuperscript{2} & Transcriptomics & Microarray & 1318 & 16 &
15.7 \\
Sarma\textsuperscript{1} & Transcriptomics & scRNA-seq & 706 & 9.8 &
10.3 \\
Nguyen\textsuperscript{3} & Proteomics & Mass Spec & 161 & 2.2 & 2.1 \\
Wang\textsuperscript{4} & Transcriptomics & Microarray & 137 & 1.9 &
1.9 \\
Bhargava\textsuperscript{5} & Proteomics & Mass Spec & 233 & 3.1 &
1.9 \\
Kovach\textsuperscript{6} & Transcriptomics & Microarray & 123 & 1.8 &
1.9 \\
Bhargava\textsuperscript{7} & Proteomics & Mass Spec & 144 & 1.9 &
1.8 \\
Morrell\textsuperscript{8} & Transcriptomics & Microarray & 155 & 1.9 &
1.7 \\
Christie\textsuperscript{9} & GWAS & Genotyping & 143 & 1.4 & 1.5 \\
Liao\textsuperscript{10} & GWAS & Genotyping & 67 & 0.7 & 0.8 \\
Sarma\textsuperscript{1} & Proteomics & Other & 60 & 0.8 & 0.7 \\
Jiang\textsuperscript{11} & Transcriptomics & scRNA-seq & 53 & 0.7 &
0.6 \\
Batra\textsuperscript{12} & Proteomics & Other & 39 & 0.6 & 0.6 \\
Bime\textsuperscript{13} & GWAS & Genotyping & 51 & 0.5 & 0.5 \\
Bos\textsuperscript{14} & Transcriptomics & Microarray & 53 & 0.7 &
0.5 \\
Chang\textsuperscript{15} & Proteomics & Mass Spec & 37 & 0.5 & 0.5 \\
Mirchandani\textsuperscript{16} & Transcriptomics & Microarray & 41 &
0.5 & 0.4 \\
Mirchandani\textsuperscript{16} & Proteomics & Mass Spec & 29 & 0.4 &
0.4 \\
Liao\textsuperscript{10} & Transcriptomics & RNA-seq & 43 & 0.4 & 0.4 \\
Dong\textsuperscript{17} & Proteomics & Mass Spec & 27 & 0.4 & 0.4 \\
Ren\textsuperscript{18} & Proteomics & Other & 17 & 0.3 & 0.3 \\
Tejera\textsuperscript{19} & GWAS & Genotyping & 19 & 0.3 & 0.3 \\
Howrylak\textsuperscript{20} & Transcriptomics & Microarray & 28 & 0.3 &
0.2 \\
Xu\textsuperscript{21} & GWAS & WES & 16 & 0.2 & 0.2 \\
Chen\textsuperscript{22} & Proteomics & Mass Spec & 16 & 0.2 & 0.2 \\
Zhang\textsuperscript{23} & Transcriptomics & RNA-seq & 20 & 0.2 &
0.2 \\
Kangelaris\textsuperscript{24} & Transcriptomics & Microarray & 15 & 0.2
& 0.2 \\
Meyer\textsuperscript{25} & GWAS & Genotyping & 10 & 0.1 & 0.1 \\
Martucci\textsuperscript{26} & Transcriptomics & Microarray & 13 & 0.1 &
0.1 \\
Zhu\textsuperscript{27} & Transcriptomics & Microarray & 14 & 0.1 &
0.1 \\
Englert\textsuperscript{28} & Transcriptomics & RNA-seq & 10 & 0.1 &
0.1 \\
Lu\textsuperscript{29} & Transcriptomics & Microarray & 12 & \textless{}
0.1 & \textless{} 0.1 \\
Scheller\textsuperscript{30} & Transcriptomics & RNA-seq & 9 &
\textless{} 0.1 & \textless{} 0.1 \\
Nick\textsuperscript{31} & Transcriptomics & Microarray & 4 &
\textless{} 0.1 & \textless{} 0.1 \\
Guillen-Guio\textsuperscript{32} & GWAS & Genotyping & 6 & \textless{}
0.1 & \textless{} 0.1 \\
Meyer\textsuperscript{33} & GWAS & Genotyping & 4 & \textless{} 0.1 &
\textless{} 0.1 \\
Dolinay\textsuperscript{34} & Transcriptomics & Microarray & 4 &
\textless{} 0.1 & \textless{} 0.1 \\
Chen\textsuperscript{35} & Proteomics & Mass Spec & 16 & \textless{} 0.1
& \textless{} 0.1 \\
Zhang\textsuperscript{36} & Transcriptomics & RNA-seq & 5 & \textless{}
0.1 & \textless{} 0.1 \\
Shortt\textsuperscript{37} & GWAS & WES & 3 & \textless{} 0.1 &
\textless{} 0.1 \\
Bowler\textsuperscript{38} & Proteomics & Mass Spec & 18 & \textless{}
0.1 & \textless{} 0.1 \\
Morrell\textsuperscript{39} & Transcriptomics & Microarray & 1 &
\textless{} 0.1 & \textless{} 0.1 \\
\end{longtable}

\begin{scriptsize}
Abbreviations: GWAS - Genome-wide association study; Mass Spec - Mass spectometry; totMS - Total MAIC score; ctotMS - Contributing total MAIC score; WES - Whole-exome sequencing. totMS and ctotMS are reported as the percentage of the sum of totMS/ctotMS for all lists included in the analysis.
\end{scriptsize}

\newpage

\textbf{Supplementary Table 2. ARDS MAIC prioritised genes found in
common by BioLitMine with \textgreater= 2 associated publications.}

\begin{longtable}[]{@{}
  >{\raggedright\arraybackslash}p{(\columnwidth - 6\tabcolsep) * \real{0.1128}}
  >{\raggedright\arraybackslash}p{(\columnwidth - 6\tabcolsep) * \real{0.1729}}
  >{\raggedright\arraybackslash}p{(\columnwidth - 6\tabcolsep) * \real{0.6015}}
  >{\raggedright\arraybackslash}p{(\columnwidth - 6\tabcolsep) * \real{0.1128}}@{}}
\toprule\noalign{}
\begin{minipage}[b]{\linewidth}\raggedright
\textbf{Gene}
\end{minipage} & \begin{minipage}[b]{\linewidth}\raggedright
\textbf{Publication count}
\end{minipage} & \begin{minipage}[b]{\linewidth}\raggedright
\textbf{PubMed IDs}
\end{minipage} & \begin{minipage}[b]{\linewidth}\raggedright
\textbf{MAIC rank}
\end{minipage} \\
\midrule\noalign{}
\endhead
\bottomrule\noalign{}
\endlastfoot
TGFB1 & 8 & 30395619, 29083412, 28188225, 27309347, 22034170, 20142324,
16100012, 12654639 & 225 \\
VEGFA & 8 & 24356493, 23542734, 21797753, 19543148, 19349383, 17289863,
15920019, 15741444 & 320 \\
IL10 & 8 & 32217834, 31936183, 30280795, 28432351, 22033829, 21138342,
18242340, 16585075 & 1268 \\
SFTPB & 6 & 21128671, 18679120, 16100012, 15190959, 14718442, 12490037 &
177 \\
IL17A & 6 & 34239039, 32795834, 32651218, 30655311, 26709006, 26002979 &
1294 \\
PI3 & 5 & 28187039, 24617927, 19251943, 19197381, 18203972 & 2 \\
CXCL8 & 5 & 22897124, 22080750, 21348591, 17498967, 14729508 & 3 \\
IL6 & 5 & 34757857, 33250487, 32826331, 31261506, 18593632 & 144 \\
TNF & 5 & 31261506, 22507624, 21784970, 17034639, 16135717 & 651 \\
NAMPT & 4 & 24821571, 24053186, 18486613, 17392604 & 58 \\
IL1RN & 4 & 30095747, 29943912, 23449693, 18838927 & 175 \\
SCGB1A1 & 4 & 32787812, 28548310, 18521628, 16215398 & 187 \\
NPPB & 4 & 28322314, 26359292, 21696613, 19830720 & 1239 \\
HGF & 3 & 18065658, 17702746, 11943656 & 343 \\
IL33 & 3 & 33936076, 31147742, 23000728 & 385 \\
CXCL10 & 3 & 31651197, 23542734, 23144331 & 671 \\
S100A12 & 2 & 26274928, 24887223 & 5 \\
MUC1 & 2 & 21418654, 17565019 & 69 \\
PLAU & 2 & 23064953, 17994220 & 244 \\
EPAS1 & 2 & 28613249, 25574837 & 425 \\
FASLG & 2 & 30385692, 12414525 & 503 \\
EDN1 & 2 & 27765761, 17875064 & 643 \\
AKT1 & 2 & 27607575, 15961723 & 950 \\
MMP8 & 2 & 24651234, 15187163 & 1223 \\
\end{longtable}

\newpage

\textbf{Supplementary Table 3. ARDS susceptibility gene list information
content and contribution.}

\begin{longtable}[]{@{}
  >{\raggedright\arraybackslash}p{(\columnwidth - 10\tabcolsep) * \real{0.2719}}
  >{\raggedright\arraybackslash}p{(\columnwidth - 10\tabcolsep) * \real{0.1491}}
  >{\raggedright\arraybackslash}p{(\columnwidth - 10\tabcolsep) * \real{0.1228}}
  >{\raggedright\arraybackslash}p{(\columnwidth - 10\tabcolsep) * \real{0.1140}}
  >{\raggedright\arraybackslash}p{(\columnwidth - 10\tabcolsep) * \real{0.1667}}
  >{\raggedright\arraybackslash}p{(\columnwidth - 10\tabcolsep) * \real{0.1754}}@{}}
\toprule\noalign{}
\begin{minipage}[b]{\linewidth}\raggedright
\textbf{Study}
\end{minipage} & \begin{minipage}[b]{\linewidth}\raggedright
\textbf{Method}
\end{minipage} & \begin{minipage}[b]{\linewidth}\raggedright
\textbf{Category}
\end{minipage} & \begin{minipage}[b]{\linewidth}\raggedright
\textbf{N genes}
\end{minipage} & \begin{minipage}[b]{\linewidth}\raggedright
\textbf{totMS} (\% sum)
\end{minipage} & \begin{minipage}[b]{\linewidth}\raggedright
\textbf{ctotMS} (\% sum)
\end{minipage} \\
\midrule\noalign{}
\endhead
\bottomrule\noalign{}
\endlastfoot
Juss\textsuperscript{2} & Transcriptomics & Microarray & 1318 & 54.7 &
54.7 \\
Nguyen\textsuperscript{3} & Proteomics & Mass Spec & 161 & 8.1 & 7.7 \\
Christie\textsuperscript{9} & GWAS & Genotyping & 143 & 6 & 6.3 \\
Kovach\textsuperscript{6} & Transcriptomics & Microarray & 123 & 5.8 &
6.1 \\
Wang\textsuperscript{4} & Transcriptomics & Microarray & 137 & 5.8 &
6 \\
Jiang\textsuperscript{11} & Transcriptomics & scRNA-seq & 53 & 2.9 &
3 \\
Bime\textsuperscript{13} & GWAS & Genotyping & 51 & 2.2 & 2.3 \\
Mirchandani\textsuperscript{16} & Transcriptomics & Microarray & 41 &
1.7 & 1.6 \\
Chang\textsuperscript{15} & Proteomics & Mass Spec & 37 & 1.9 & 1.5 \\
Mirchandani\textsuperscript{16} & Proteomics & Mass Spec & 29 & 1.4 &
1.3 \\
Howrylak\textsuperscript{20} & Transcriptomics & Microarray & 28 & 1.2 &
1.3 \\
Ren\textsuperscript{18} & Proteomics & Other & 17 & 1 & 1.1 \\
Tejera\textsuperscript{19} & GWAS & Genotyping & 19 & 0.9 & 1 \\
Chen\textsuperscript{35} & Proteomics & Mass Spec & 16 & 0.9 & 0.9 \\
Zhang\textsuperscript{23} & Transcriptomics & RNA-seq & 20 & 0.8 &
0.9 \\
Zhu\textsuperscript{27} & Transcriptomics & Microarray & 14 & 0.6 &
0.6 \\
Kangelaris\textsuperscript{24} & Transcriptomics & Microarray & 15 & 0.7
& 0.6 \\
Englert\textsuperscript{28} & Transcriptomics & RNA-seq & 10 & 0.6 &
0.6 \\
Lu\textsuperscript{29} & Transcriptomics & Microarray & 12 & 0.5 &
0.5 \\
Meyer\textsuperscript{25} & GWAS & Genotyping & 10 & 0.4 & 0.4 \\
Bowler\textsuperscript{38} & Proteomics & Mass Spec & 18 & 0.9 & 0.4 \\
Scheller\textsuperscript{30} & Transcriptomics & RNA-seq & 9 & 0.4 &
0.3 \\
Guillen-Guio\textsuperscript{32} & GWAS & Genotyping & 6 & 0.2 & 0.2 \\
Zhang\textsuperscript{36} & Transcriptomics & RNA-seq & 5 & 0.2 & 0.2 \\
Dolinay\textsuperscript{34} & Transcriptomics & Microarray & 4 & 0.2 &
0.2 \\
Shortt\textsuperscript{37} & GWAS & WES & 3 & 0.1 & 0.1 \\
Meyer\textsuperscript{33} & GWAS & Genotyping & 4 & \textless{} 0.1 &
0.1 \\
Morrell\textsuperscript{39} & Transcriptomics & Microarray & 1 &
\textless{} 0.1 & \textless{} 0.1 \\
\end{longtable}

\begin{scriptsize}
Abbreviations: GWAS - Genome-wide association study; Mass Spec - Mass spectometry; totMS - Total MAIC score; ctotMS - Contributing total MAIC score; WES - Whole-exome sequencing. totMS and ctotMS are reported as the percentage of the sum of totMS/ctotMS for all lists included in the analysis.
\end{scriptsize}

\newpage

\textbf{Supplementary Table 4. ARDS survival/severity gene list
information content and contribution.}

\begin{longtable}[]{@{}
  >{\raggedright\arraybackslash}p{(\columnwidth - 10\tabcolsep) * \real{0.2719}}
  >{\raggedright\arraybackslash}p{(\columnwidth - 10\tabcolsep) * \real{0.1491}}
  >{\raggedright\arraybackslash}p{(\columnwidth - 10\tabcolsep) * \real{0.1228}}
  >{\raggedright\arraybackslash}p{(\columnwidth - 10\tabcolsep) * \real{0.1140}}
  >{\raggedright\arraybackslash}p{(\columnwidth - 10\tabcolsep) * \real{0.1667}}
  >{\raggedright\arraybackslash}p{(\columnwidth - 10\tabcolsep) * \real{0.1754}}@{}}
\toprule\noalign{}
\begin{minipage}[b]{\linewidth}\raggedright
\textbf{Study}
\end{minipage} & \begin{minipage}[b]{\linewidth}\raggedright
\textbf{Method}
\end{minipage} & \begin{minipage}[b]{\linewidth}\raggedright
\textbf{Category}
\end{minipage} & \begin{minipage}[b]{\linewidth}\raggedright
\textbf{N genes}
\end{minipage} & \begin{minipage}[b]{\linewidth}\raggedright
\textbf{totMS} (\% sum)
\end{minipage} & \begin{minipage}[b]{\linewidth}\raggedright
\textbf{ctotMS} (\% sum)
\end{minipage} \\
\midrule\noalign{}
\endhead
\bottomrule\noalign{}
\endlastfoot
Bhargava\textsuperscript{7} & Proteomics & Mass Spec & 144 & 30.4 &
30.3 \\
Morrell\textsuperscript{8} & Transcriptomics & Microarray & 155 & 29.7 &
29.7 \\
Liao\textsuperscript{10} & GWAS & Genotyping & 67 & 12.9 & 13 \\
Batra\textsuperscript{12} & Proteomics & Other & 39 & 9.4 & 9.4 \\
Liao\textsuperscript{10} & Transcriptomics & RNA-seq & 43 & 8.5 & 8.5 \\
Xu\textsuperscript{21} & GWAS & WES & 16 & 3.5 & 3.5 \\
Chen\textsuperscript{22} & Proteomics & Mass Spec & 16 & 3.4 & 3.4 \\
Lu\textsuperscript{29} & Transcriptomics & Microarray & 12 & 2.2 &
2.2 \\
\end{longtable}

\begin{scriptsize}
Abbreviations: GWAS - Genome-wide association study; Mass Spec - Mass spectometry; totMS - Total MAIC score; ctotMS - Contributing total MAIC score; WES - Whole-exome sequencing. totMS and ctotMS are reported as the percentage of the sum of totMS/ctotMS for all lists included in the analysis.
\end{scriptsize}

\newpage

\subsection{Supplementary Data}\label{supplementary-data}

Supplementary Data Files 1-8

\newpage

\textbf{Supplementary Data File 1. Raw gene list input to MAIC}.

\url{https://github.com/JonathanEMillar/ards_maic_manuscript/Supplementary_Data_File_1.csv}

\textbf{Supplementary Data File 2. MAIC output - overall}.

\url{https://github.com/JonathanEMillar/ards_maic_manuscript/Supplementary_Data_File_2.csv}

\textbf{Supplementary Data File 3. BioLitMine and ARDS Database of Genes
results}.

\url{https://github.com/JonathanEMillar/ards_maic_manuscript/Supplementary_Data_File_3.csv}

\textbf{Supplementary Data File 4. MAIC output - ARDS vs.~non-ARDS
controls sub-group}.

\url{https://github.com/JonathanEMillar/ards_maic_manuscript/Supplementary_Data_File_4.csv}

\textbf{Supplementary Data File 5. MAIC output - survival sub-group}.

\url{https://github.com/JonathanEMillar/ards_maic_manuscript/Supplementary_Data_File_5.csv}

\textbf{Supplementary Data File 6. Functional enrichment results -
overall}.

\url{https://github.com/JonathanEMillar/ards_maic_manuscript/Supplementary_Data_File_6.csv}

\textbf{Supplementary Data File 7. Functional enrichment results - ARDS
vs.~non-ARDS controls sub-group}.

\url{https://github.com/JonathanEMillar/ards_maic_manuscript/Supplementary_Data_File_7.csv}

\textbf{Supplementary Data File 8. Functional enrichment results -
survival sub-group}.

\url{https://github.com/JonathanEMillar/ards_maic_manuscript/Supplementary_Data_File_8.csv}

\newpage

\subsection{References}\label{references}

\phantomsection\label{refs}
\begin{CSLReferences}{0}{0}
\bibitem[\citeproctext]{ref-Sarma2022}
\CSLLeftMargin{1. }%
\CSLRightInline{Sarma, A. \emph{et al.} Hyperinflammatory {ARDS} is
characterized by interferon-stimulated gene expression, t-cell
activation, and an altered metatranscriptome in tracheal aspirates.
\emph{bioRxiv} (2022).}

\bibitem[\citeproctext]{ref-Juss2016}
\CSLLeftMargin{2. }%
\CSLRightInline{Juss, J. K. \emph{et al.} Acute respiratory distress
syndrome neutrophils have a distinct phenotype and are resistant to
phosphoinositide 3-kinase inhibition. \emph{Am. J. Respir. Crit. Care
Med.} \textbf{194}, 961--973 (2016).}

\bibitem[\citeproctext]{ref-Nguyen2013}
\CSLLeftMargin{3. }%
\CSLRightInline{Nguyen, E. V. \emph{et al.} Proteomic profiling of
bronchoalveolar lavage fluid in critically ill patients with
ventilator-associated pneumonia. \emph{PLoS One} \textbf{8}, e58782
(2013).}

\bibitem[\citeproctext]{ref-Wang2008}
\CSLLeftMargin{4. }%
\CSLRightInline{Wang, Z., Beach, D., Su, L., Zhai, R. \& Christiani, D.
C. A genome-wide expression analysis in blood identifies pre-elafin as a
biomarker in {ARDS}. \emph{Am. J. Respir. Cell Mol. Biol.} \textbf{38},
724--732 (2008).}

\bibitem[\citeproctext]{ref-Bhargava2014}
\CSLLeftMargin{5. }%
\CSLRightInline{Bhargava, M. \emph{et al.} Proteomic profiles in acute
respiratory distress syndrome differentiates survivors from
non-survivors. \emph{PLoS One} \textbf{9}, e109713 (2014).}

\bibitem[\citeproctext]{ref-Kovach2015}
\CSLLeftMargin{6. }%
\CSLRightInline{Kovach, M. A. \emph{et al.} Microarray analysis
identifies {IL-1} receptor type 2 as a novel candidate biomarker in
patients with acute respiratory distress syndrome. \emph{Respir. Res.}
\textbf{16}, 29 (2015).}

\bibitem[\citeproctext]{ref-Bhargava2017}
\CSLLeftMargin{7. }%
\CSLRightInline{Bhargava, M. \emph{et al.} Bronchoalveolar lavage fluid
protein expression in acute respiratory distress syndrome provides
insights into pathways activated in subjects with different outcomes.
\emph{Sci. Rep.} \textbf{7}, 7464 (2017).}

\bibitem[\citeproctext]{ref-Morrell2019}
\CSLLeftMargin{8. }%
\CSLRightInline{Morrell, E. D. \emph{et al.} Alveolar macrophage
transcriptional programs are associated with outcomes in acute
respiratory distress syndrome. \emph{Am. J. Respir. Crit. Care Med.}
\textbf{200}, 732--741 (2019).}

\bibitem[\citeproctext]{ref-Christie2012}
\CSLLeftMargin{9. }%
\CSLRightInline{Christie, J. D. \emph{et al.} Genome wide association
identifies {PPFIA1} as a candidate gene for acute lung injury risk
following major trauma. \emph{PLoS One} \textbf{7}, e28268 (2012).}

\bibitem[\citeproctext]{ref-Liao2021}
\CSLLeftMargin{10. }%
\CSLRightInline{Liao, S. Y. \emph{et al.} Identification of early and
intermediate biomarkers for {ARDS} mortality by multi-omic approaches.
\emph{Sci. Rep.} \textbf{11}, 18874 (2021).}

\bibitem[\citeproctext]{ref-Jiang2020}
\CSLLeftMargin{11. }%
\CSLRightInline{Jiang, Y. \emph{et al.} Single cell {RNA} sequencing
identifies an early monocyte gene signature in acute respiratory
distress syndrome. \emph{JCI Insight} \textbf{5}, (2020).}

\bibitem[\citeproctext]{ref-Batra2022}
\CSLLeftMargin{12. }%
\CSLRightInline{Batra, R. \emph{et al.} Multi-omic comparative analysis
of {COVID-19} and bacterial sepsis-induced {ARDS}. \emph{PLoS Pathog.}
\textbf{18}, e1010819 (2022).}

\bibitem[\citeproctext]{ref-Bime2018}
\CSLLeftMargin{13. }%
\CSLRightInline{Bime, C. \emph{et al.} Genome-wide association study in
african americans with acute respiratory distress syndrome identifies
the selectin {P} ligand gene as a risk factor. \emph{Am. J. Respir.
Crit. Care Med.} \textbf{197}, 1421--1432 (2018).}

\bibitem[\citeproctext]{ref-Bos2019}
\CSLLeftMargin{14. }%
\CSLRightInline{Bos, L. D. J. \emph{et al.} Understanding heterogeneity
in biologic phenotypes of acute respiratory distress syndrome by
leukocyte expression profiles. \emph{Am. J. Respir. Crit. Care Med.}
\textbf{200}, 42--50 (2019).}

\bibitem[\citeproctext]{ref-Chang2008}
\CSLLeftMargin{15. }%
\CSLRightInline{Chang, D. W. \emph{et al.} Proteomic and computational
analysis of bronchoalveolar proteins during the course of the acute
respiratory distress syndrome. \emph{Am. J. Respir. Crit. Care Med.}
\textbf{178}, 701--709 (2008).}

\bibitem[\citeproctext]{ref-Mirchandani2022}
\CSLLeftMargin{16. }%
\CSLRightInline{Mirchandani, A. S. \emph{et al.} Hypoxia shapes the
immune landscape in lung injury and promotes the persistence of
inflammation. \emph{Nat. Immunol.} \textbf{23}, 927--939 (2022).}

\bibitem[\citeproctext]{ref-Dong2013}
\CSLLeftMargin{17. }%
\CSLRightInline{Dong, H. \emph{et al.} Comparative analysis of the
alveolar macrophage proteome in {ALI/ARDS} patients between the
exudative phase and recovery phase. \emph{BMC Immunol.} \textbf{14}, 25
(2013).}

\bibitem[\citeproctext]{ref-Ren2016}
\CSLLeftMargin{18. }%
\CSLRightInline{Ren, S. \emph{et al.} Deleted in malignant brain tumors
1 protein is a potential biomarker of acute respiratory distress
syndrome induced by pneumonia. \emph{Biochem. Biophys. Res. Commun.}
\textbf{478}, 1344--1349 (2016).}

\bibitem[\citeproctext]{ref-Tejera2012}
\CSLLeftMargin{19. }%
\CSLRightInline{Tejera, P. \emph{et al.} Distinct and replicable genetic
risk factors for acute respiratory distress syndrome of pulmonary or
extrapulmonary origin. \emph{J. Med. Genet.} \textbf{49}, 671--680
(2012).}

\bibitem[\citeproctext]{ref-Howrylak2009}
\CSLLeftMargin{20. }%
\CSLRightInline{Howrylak, J. A. \emph{et al.} Discovery of the gene
signature for acute lung injury in patients with sepsis. \emph{Physiol.
Genomics} \textbf{37}, 133--139 (2009).}

\bibitem[\citeproctext]{ref-Xu2021}
\CSLLeftMargin{21. }%
\CSLRightInline{Xu, J.-Y. \emph{et al.} Nucleotide polymorphism in
{ARDS} outcome: A whole exome sequencing association study. \emph{Ann.
Transl. Med.} \textbf{9}, 780 (2021).}

\bibitem[\citeproctext]{ref-Chen2016}
\CSLLeftMargin{22. }%
\CSLRightInline{Chen, C., Shi, L., Li, Y., Wang, X. \& Yang, S.
Disease-specific dynamic biomarkers selected by integrating inflammatory
mediators with clinical informatics in {ARDS} patients with severe
pneumonia. \emph{Cell Biol. Toxicol.} \textbf{32}, 169--184 (2016).}

\bibitem[\citeproctext]{ref-Zhang2022}
\CSLLeftMargin{23. }%
\CSLRightInline{Zhang, C. \emph{et al.} Differential expression profile
of plasma exosomal {microRNAs} in acute type a aortic dissection with
acute lung injury. \emph{Sci. Rep.} \textbf{12}, 11667 (2022).}

\bibitem[\citeproctext]{ref-Kangelaris2015}
\CSLLeftMargin{24. }%
\CSLRightInline{Kangelaris, K. N. \emph{et al.} Increased expression of
neutrophil-related genes in patients with early sepsis-induced {ARDS}.
\emph{Am. J. Physiol. Lung Cell. Mol. Physiol.} \textbf{308}, L1102--13
(2015).}

\bibitem[\citeproctext]{ref-Meyer2013}
\CSLLeftMargin{25. }%
\CSLRightInline{Meyer, N. J. \emph{et al.} {IL1RN} coding variant is
associated with lower risk of acute respiratory distress syndrome and
increased plasma {IL-1} receptor antagonist. \emph{Am. J. Respir. Crit.
Care Med.} \textbf{187}, 950--959 (2013).}

\bibitem[\citeproctext]{ref-Martucci2020}
\CSLLeftMargin{26. }%
\CSLRightInline{Martucci, G. \emph{et al.} Identification of a
circulating {miRNA} signature to stratify acute respiratory distress
syndrome patients. \emph{J. Pers. Med.} \textbf{11}, 15 (2020).}

\bibitem[\citeproctext]{ref-Zhu2017}
\CSLLeftMargin{27. }%
\CSLRightInline{Zhu, Z. \emph{et al.} Whole blood {microRNA} markers are
associated with acute respiratory distress syndrome. \emph{Intensive
Care Med. Exp.} \textbf{5}, 38 (2017).}

\bibitem[\citeproctext]{ref-Englert2019}
\CSLLeftMargin{28. }%
\CSLRightInline{Englert, J. A. \emph{et al.} Whole blood {RNA}
sequencing reveals a unique transcriptomic profile in patients with
{ARDS} following hematopoietic stem cell transplantation. \emph{Respir.
Res.} \textbf{20}, 15 (2019).}

\bibitem[\citeproctext]{ref-Lu2017}
\CSLLeftMargin{29. }%
\CSLRightInline{Lu, X.-G. \emph{et al.} Circulating {miRNAs} as
biomarkers for severe acute pancreatitis associated with acute lung
injury. \emph{World J. Gastroenterol.} \textbf{23}, 7440--7449 (2017).}

\bibitem[\citeproctext]{ref-Scheller2019}
\CSLLeftMargin{30. }%
\CSLRightInline{Scheller, N. \emph{et al.} Proviral {MicroRNAs} detected
in extracellular vesicles from bronchoalveolar lavage fluid of patients
with influenza virus-induced acute respiratory distress syndrome.
\emph{J. Infect. Dis.} \textbf{219}, 540--543 (2019).}

\bibitem[\citeproctext]{ref-Nick2016}
\CSLLeftMargin{31. }%
\CSLRightInline{Nick, J. A. \emph{et al.} Extremes of
interferon-stimulated gene expression associate with worse outcomes in
the acute respiratory distress syndrome. \emph{PLoS One} \textbf{11},
e0162490 (2016).}

\bibitem[\citeproctext]{ref-GuillenGuio2020}
\CSLLeftMargin{32. }%
\CSLRightInline{Guillen-Guio, B. \emph{et al.} Sepsis-associated acute
respiratory distress syndrome in individuals of european ancestry: A
genome-wide association study. \emph{Lancet Respir. Med.} \textbf{8},
258--266 (2020).}

\bibitem[\citeproctext]{ref-Meyer2011}
\CSLLeftMargin{33. }%
\CSLRightInline{Meyer, N. J. \emph{et al.} {ANGPT2} genetic variant is
associated with trauma-associated acute lung injury and altered plasma
angiopoietin-2 isoform ratio. \emph{Am. J. Respir. Crit. Care Med.}
\textbf{183}, 1344--1353 (2011).}

\bibitem[\citeproctext]{ref-Dolinay2012}
\CSLLeftMargin{34. }%
\CSLRightInline{Dolinay, T. \emph{et al.} Inflammasome-regulated
cytokines are critical mediators of acute lung injury. \emph{Am. J.
Respir. Crit. Care Med.} \textbf{185}, 1225--1234 (2012).}

\bibitem[\citeproctext]{ref-Chen2013}
\CSLLeftMargin{35. }%
\CSLRightInline{Chen, X., Shan, Q., Jiang, L., Zhu, B. \& Xi, X.
Quantitative proteomic analysis by {iTRAQ} for identification of
candidate biomarkers in plasma from acute respiratory distress syndrome
patients. \emph{Biochem. Biophys. Res. Commun.} \textbf{441}, 1--6
(2013).}

\bibitem[\citeproctext]{ref-Zhang2021}
\CSLLeftMargin{36. }%
\CSLRightInline{Zhang, S. \emph{et al.} miR-584 and miR-146 are
candidate biomarkers for acute respiratory distress syndrome. \emph{Exp.
Ther. Med.} \textbf{21}, 445 (2021).}

\bibitem[\citeproctext]{ref-Shortt2014}
\CSLLeftMargin{37. }%
\CSLRightInline{Shortt, K. \emph{et al.} Identification of novel single
nucleotide polymorphisms associated with acute respiratory distress
syndrome by exome-seq. \emph{PLoS One} \textbf{9}, e111953 (2014).}

\bibitem[\citeproctext]{ref-Bowler2004}
\CSLLeftMargin{38. }%
\CSLRightInline{Bowler, R. P. \emph{et al.} Proteomic analysis of
pulmonary edema fluid and plasma in patients with acute lung injury.
\emph{Am. J. Physiol. Lung Cell. Mol. Physiol.} \textbf{286}, L1095--104
(2004).}

\bibitem[\citeproctext]{ref-Morrell2018}
\CSLLeftMargin{39. }%
\CSLRightInline{Morrell, E. D. \emph{et al.} Cytometry {TOF} identifies
alveolar macrophage subtypes in acute respiratory distress syndrome.
\emph{JCI Insight} \textbf{3}, (2018).}

\end{CSLReferences}



\end{document}
